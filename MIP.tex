
\documentclass[a4paper,USenglish]{lipics-v2018}

\usepackage{amsmath,amssymb,amsfonts,mathtools}
\usepackage{cases}

\title{Link weight, waypoint Optimization}

\begin{document}

\maketitle

\section{Single Waypoint}

\begin{align}
& \text{Minimize} ~L \nonumber	&&\\
% variables 
&M := \sum_{(s,t,d)_i \in \mathcal{D}} d &&
\\
&\mathcal{S} := \{(p,q) \in V^2 ~|~  p \neq q \}	&&
\\
&\begin{cases}
	S^{(s,t)_i}_{(p,q)}, x^{p}_{\ell} \in \{0,1\} \\	%\omega^{(s,t)_i}_{v}
	f^{(p)}_{\ell}, f^{(p,q)}_{v}, d^{p}_v, y_{\ell}  \geq 0
\end{cases}
&&
\begin{multlined}
	  (p,q) \in \mathcal{S}, \ell \in E, \\
	   (s,t,d)_i \in \mathcal{D}
\end{multlined}	 \label{LP:vardef}    
\\
% flow source constraints, one per  (p,q)-segment, p or q is source/terminal
%&\sum_{\ell = (p,*)} f^{(p,q)}_{\ell}  =
%	   \sum_{\hspace{-5ex}\mathrlap{ (s,t,d)_i \in \mathcal{D}, t=q}} 
%		\omega^{(s,t)_i}_{p}\cdot d +
%	   \sum_{\hspace{-5ex}\mathrlap{ (s,t,d)_i \in \mathcal{D}, p=s}} 
%		\omega^{(s,t)_i}_{q}\cdot d		
%&&	(p,q) \in \mathcal{S}	\label{LP:flowWP}
%\\
% segment flow constraints
&\sum_{\ell = (p,*)} f^{(p,q)}_{\ell}  =
\sum_{\hspace{-5ex}\mathrlap{ (s,t,d)_i \in \mathcal{D}}} 
	S^{(s,t)_i}_{(p,q)} \cdot d 
&&	 (p,q) \in \mathcal{S}	\label{LP:flowWP}
\\
%  flow conservation
&\sum_{\ell = (*,v)} f^{(p,q)}_{\ell} =
	\sum_{\hspace{-5ex}\mathrlap{\ell= (v,*)} }f^{(p,q)}_{\ell} 
&&
\begin{multlined}
	(p,q) \in \mathcal{S}, \\
	 v \not\in \{p,q\}
\end{multlined} 	\label{LP:flowcons}
\\
% force segment graph into an SR-path, i.e. connecting s_i to t_i and linear
&\sum_{\hspace{-4ex}\mathrlap{ p:(p,w) \in \mathcal{S}}} 	S^{(s,t)_i}_{(p,w)}	
	\leq \sum_{\hspace{-4ex}\mathrlap{ q:(w,q) \in \mathcal{S}}}
S^{(s,t)_i}_{(w,q) }	\leq 1
&&	 
\begin{multlined}
	(s,t,d)_i \in \mathcal{D}, \\
	w \in V, w \neq t
\end{multlined} \label{LP:segmentDegree}
\\
% force one SR-path from every source node
&\sum_{\hspace{-4ex}\mathrlap{ q:(s,q) \in \mathcal{S}}} S^{(s,t)_i}_{(s,q)} = 1
&&  (s,t,d)_i \in \mathcal{D}
\label{LP:wp_limit} 
\\
% waypoint limit
%&\sum_{v \in V, v \neq s}  \omega^{(s,t)}_v = 1
%&&  (s,t,d)_i \in \mathcal{D}
%\label{LP:wp_limit} 
%\\
% waypoint limit
&\sum_{\hspace{-4ex}\mathrlap{(p,q) \in \mathcal{S}}} S^{(s,t)_i}_{(p,q)}  \leq W+1
&&  (s,t,d)_i \in \mathcal{D}
\label{LP:wp_limit} 
\\
% capacities
& \sum_{\hspace{-3ex}\mathrlap{(p,q) \in \mathcal{S}}} f^{(p,q)}_{\ell} \leq L \cdot C_{\ell}
&&  \ell \in E
\\
% shortest path trees
&   f^{(p,q)}_{\ell} \leq M \cdot x^{(q)}_{\ell}
&&  \ell \in E, (p,q) \in \mathcal{S}
\label{LP:SPTree} 
\\
% equal split
&
\begin{cases*}
f^{(s,t)}_{\ell} \leq f^{(s,t)}_{v}	\\
f^{(s,t)}_{v} - f^{(s,t)}_{\ell} \leq M(1 - x^{t}_{\ell}) 
\end{cases*}	
&&  v,s,t \in V, \ell = (v,*)
\label{LP:equalSplit} 
\\ 
%%waypoint setting
%&\begin{cases}
%	h^{(s,t)}_{v,\ell} \leq (1-x^{(v)}_{\ell})	\\
%	h^{(s,t)}_{v,\ell} \leq	w^{(s,t)}_v 
%\end{cases}
%&&  v \in V, \ell = (v,*), (s,t,d) \in \mathcal{D}
% \label{LP:equalSplit3} 
%\\
% shortest paths, weight setting
&\begin{cases*}
d^{t}_u \leq d^{t}_v + y_{\ell} \\
d^{t}_v - d^{t}_u + y_{\ell} \leq M(1 - x^t_{\ell})	\\
1 - x^t_{\ell} \leq M(d^{t}_v - d^{t}_u + y_{\ell})
\end{cases*}
&&  \ell = (u,v),  t \in V
\label{LP:weights}
\end{align}


\section{General MIP}
\begin{itemize}
\item	
	$x^{(t)}_{\ell}$:	binary variables indicating whether link $\ell$ is on a shortest path to node $t$.
\item	
	$f^{(t)}_{\ell}$: the fractional amount of $(s,t)$-flow on link $\ell$.
\item	
	$f^{(t)}_{v}$:  the amount of $(s,t)$-flow on every outgoing link from node $v$.
\item
	$\omega^{(s,t)}_{w}$: binary variable indicating whether node $w$ is a waypoint on the $(s,t)$-flow. 
\item $\tau^{(s,t)}_{s,w}, \tau^{(s,t)}_{w,t}$: binary variables indicating active status of segments $(s,w)$ and $(w,t)$.
\item
	$d^{t}_v$: shortest path weight from $v$ to $t$.
\item	
	$y_{\ell}$: link weights.
\item	
	$h^{(s,t)}_{(s,w)}, h^{(s,t)}_{(w,t)}$: binary (helper) variables.
\item
	$W$: maximum  number of waypoints allowed per flow (constant).
\item
	$M$: a constant large enough.	
\end{itemize}

\begin{align}
& \text{Minimize} ~L \nonumber	&&\\
% variables 
&M := \sum_{ (s,t,d) \in \mathcal{D}} d &&\\
&\begin{cases}
	h^{(s,t)}_{(s,w)}, h^{(s,t)}_{(w,t)}  \in \{0,1\} \\
	\tau^{(s,t)}_{(s,w)}, \tau^{(s,t)}_{(w,t)},
	\omega^{(s,t)}_{w},
	x^{(t)}_{\ell} 
 \in \{0,1\} \\
 f^{(t)}_{\ell}, f^{(s,t)}_{v}, d^{t}_v, y_{\ell}  \geq 0
\end{cases}
&&  s,t,w \in V, s \neq t, \ell \in E \label{LP:vardef}    
\\
% left segment
&\sum_{\ell = (*,q)} f^{(p,q)}_{\ell} 
= \sum_{ s,t\in V, s\neq t} \tau^{(s,t)}_{(p,q)}
%\sum_{\ell= (v,*)}f^{(s,t)}_{\ell}
&&  p,q \in V
\label{LP:flowsink1}
 \\
 % right segment
 &\sum_{\ell = (v,*)} f^{(p,q)}_{\ell} = 
 \sum_{\ell = (*,v)} f^{(p,q)}_{\ell} 
 &&  p,q,v \in V, v \not\in \{p,q\}
 \label{LP:flowcons1}
 \\
 % flow source
% &\sum_{\ell = (*,w)} f^{(w,t)}_{\ell} - \sum_{\ell= (w,*)}f^{(w,t)}_{\ell}
% 	= \sum_{  t \in V, \neq w} - \tau^{(w,t)}_{(w,t)}
% &&  w \in V
% \label{LP:flowsource}
% \\
 & 
 \begin{cases}
	 \omega^{(s,t)}_{t} = 1 \\
	 \tau^{(s,t)}_{(s,t)} = d
 \end{cases}
 &&  (s,t,d) \in \mathcal{D}
 \label{LP:wp_t}
 \\
 % segment activation
 &
 \begin{cases*}
% 	\omega^{(s,t)}_{w} -\tau^{(s,t)}_{(s,w)} \leq h^{(s,t)}_{(s,w)}\\
% 	 \omega^{(s,t)}_{w} -\tau^{(s,t)}_{(w,t)} \leq h^{(s,t)}_{(w,t)}	\\
%	h^{(s,t)}_{(s,w)} \leq  \sum_{w' \in V} \omega^{(s,w)}_{w'}	\\
%    h^{(s,t)}_{(w,t)}\leq  \sum_{w' \in V} \omega^{(w,t)}_{w'}	
    \omega^{(s,t)}_{w} \cdot (1-\tau^{(s,t)}_{(s,w)}) \leq  \sum_{w' \in V} \omega^{(s,w)}_{w'}	\\
    \omega^{(s,t)}_{w} \cdot (1-\tau^{(s,t)}_{(w,t)})  \leq  \sum_{w' \in V} \omega^{(w,t)}_{w'}	    
 \end{cases*}	 
 &&  s,t,w \in V, s\neq t
 \label{LP:segments}
  \\
 &
 \begin{cases*}
  \omega^{(s,t)}_{w} \cdot \tau^{(s,t)}_{(s,t)} \leq \tau^{(s,t)}_{(s,w)}	\\
 \omega^{(s,t)}_{w}  \cdot \tau^{(s,t)}_{(s,t)} \leq  \tau^{(s,t)}_{(w,t)}\\
 \sum_{ v\in V} \tau^{(s,v)}_{(s,w)} \leq \tau^{(s,w)}_{(s,w)} \\
   \sum_{ v\in V} \tau^{(v,t)}_{(w,t)} \leq  \tau^{(w,t)}_{(w,t)} 
 \end{cases*}
 &&  s,t,w \in V, s\neq t
 \\
  % waypoint limit
 &\sum_{v \in V}  \omega^{(s,t)}_v \leq W
 &&  (s,t,d) \in \mathcal{D}
 \label{LP:wp_limit} 
 \\
 % capacities
 & \sum_{ s,t \in V} f^{(s,t)}_{\ell} \leq L \cdot C_{\ell}
 &&  \ell \in E
 \\
% % flow through \ell
% &  f^{(s,t)}_{\ell}  = \sum_{(s,t,d) \in \mathcal{D}} f^{(s,t)}_{\ell} 
% && \ell \in E, t \in V: \exists(s,t,d) \in \mathcal{D}
% \label{LP:ellFlow} 
% \\
  % shortest path trees
&   f^{(s,t)}_{\ell} \leq M \cdot x^{(t)}_{\ell}
&&  \ell \in E, s,t \in V
 \label{LP:SPTree} 
 \\
% equal split
&
\begin{cases*}
	f^{(s,t)}_{\ell} \leq f^{(s,t)}_{v}	\\
	f^{(s,t)}_{v} - f^{(s,t)}_{\ell} \leq M(1 - x^{t}_{\ell}) 
\end{cases*}	
&&  v,s,t \in V, \ell = (v,*)
 \label{LP:equalSplit} 
\\ 
%%waypoint setting
%&\begin{cases}
%	h^{(s,t)}_{v,\ell} \leq (1-x^{(v)}_{\ell})	\\
%	h^{(s,t)}_{v,\ell} \leq	w^{(s,t)}_v 
%\end{cases}
%&&  v \in V, \ell = (v,*), (s,t,d) \in \mathcal{D}
% \label{LP:equalSplit3} 
%\\
% shortest paths, weight setting
&\begin{cases*}
	d^{t}_u \leq d^{t}_v + y_{\ell} \\
	d^{t}_v - d^{t}_u + y_{\ell} \leq M(1 - x^t_{\ell})	\\
	1 - x^t_{\ell} \leq M(d^{t}_v - d^{t}_u + y_{\ell})
\end{cases*}
&&  \ell = (u,v),  t \in V
 \label{LP:weights} 
\end{align}

Constraints (\ref{LP:flowdest}) impose flow conservation for every destination (sink node) $t$
and as well as for the ordinary  nodes which must preserve the flow.
The right hand side of these constraints is non-zero only if there is a demand from (to) the corresponding source (destination). For original sources and destinations, i.e., those given as input, 
the r.h.s is always non-zero and is the total  demand from (to) that node.
This is ensured by constraints (\ref{LP:wp_t}) and  (\ref{LP:segments}).
 Constraints (\ref{LP:wp_t}) designate the destination of each demand as its (default) waypoint.
 A waypoint $w$  on a segment $(s,t)$ is indicated by $\omega^{(s,t)}_{w}$.
Constraints at (\ref{LP:segments}) ensure that if for a segment $(s,t)$
a waypoint $w$ is set (i.e., $\omega^{(s,t)}_{w}=1$),
then the two (sub-)segments $(s,w)$ and $(w,t)$ are active.
That is,
$\tau^{(s,t)}_{(s,w)} = \tau^{(s,t)}_{(w,t)} = 1$.
That is, one flow must run from $s$ to $w$ and another from $w$ to $w$, both with size equal to the value $d$ of the demand $(s,t,d) \in \mathcal{D}$.
A segment is active 
\end{document}
